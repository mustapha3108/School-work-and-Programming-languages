\documentclass[12pt,a4paper]{article}

\usepackage{graphicx}
\usepackage{float}
\usepackage{booktabs}
\usepackage{geometry}
\geometry{
	left=2.5cm,
	right=2.5cm,
	top=2.5cm,
	bottom=2.5cm,
}

\begin{document}
	
	% ------------------------------------------------------
	%                  COVER PAGE
	% ------------------------------------------------------
	\begin{titlepage}
		\begin{center}
			
			\vspace*{2cm}
			
			{\Huge \textbf{Université de Sciences et Téchnologies Houari Boumedien}}\\[1.0cm]
			
			{\Large Faculty d'Informatique}\\[0.2cm]
			{\Large Master 1 RSD}\\[2.0cm]
			
			{\huge \textbf{TP ASGBD}}\\[0.5cm]
		\end{center}	
			\textbf{Belmouloud Mustapha Abdellah 212131092524}\\[0.2cm]
			\textbf{Amine Chouial 222231579814} \\[0.2cm]
			\textbf{Merazi Wail 212133045973} \\[0.2cm]
			\textbf{Mohamed Amine Khoucha 212131052028} \\[0.2cm]
			\vfill
			
			{\large \today}
			
	\end{titlepage}
	
	\section{TP1}
	\begin{enumerate}
		\item
		 	create tablespace belmouloud\_tbs datafile '/opt/oracle/oradata/XE/belmouloud\_tbs.dbf' SIZE 100M AUTOEXTEND ON;
			\begin{figure}[H]
				\centering
				\includegraphics[width=0.9\textwidth]{/home/crow/Pictures/Screenshots/1.png}
			\end{figure}
			
			create temporary tablespace belmouloud2\_temptbs tempfile '/opt/oracle/oradata/XE/\newline belmouloud2\_temptbs.dbf' SIZE 100M AUTOEXTEND ON;
			\begin{figure}[H]
				\centering
				\includegraphics[width=0.9\textwidth]{/home/crow/Pictures/Screenshots/2.png}
			\end{figure}
		
		\item 
			create user tp1\_belmouloud identified by musmus3108 default tablespace belmouloud\_tbs temporary tablespace belmouloud2\_temptbs;
			\begin{figure}[H]
				\centering
				\includegraphics[width=0.9\textwidth]{/home/crow/Pictures/Screenshots/3.png}
			\end{figure}
		
		\item 
			GRANT ALL privileges to tp1\_belmouloud;
			\begin{figure}[H]
				\centering
				\includegraphics[width=0.9\textwidth]{/home/crow/Pictures/Screenshots/4.png}
			\end{figure}
		
		\item 
			les tableaux proposés: \newline
			
			user(id primary key, name) \newline
			
			sellers(id primary key, name) \newline
			
			products(id primary key, name, seller\_id foreign key, price(entre 20 et 100), category (tshirt, hoodie, jacket)) \newline
			
			orders(product\_id foreign key, user\_id foreign key, primarykey(product\_id, use\_id)) \newline
			
			order\_comments (id primary key, product\_id foreign key, user\_id foreign key,comments) \newline
			
			seller\_reviews (id primary\_key, seller\_id foeignkey,user\_id foreignkey,rating) 
			
		\item 
			 CREATE TABLE users (
			 id NUMBER,
			 name VARCHAR2(50),
			 CONSTRAINT pk\_users PRIMARY KEY (id)
			 ); \newline
			 
			 CREATE TABLE sellers (
			 id NUMBER,
			 name VARCHAR2(50),
			 CONSTRAINT pk\_sellers PRIMARY KEY (id)
			 ); \newline
			 
			 CREATE TABLE products (
			 id NUMBER,
			 name VARCHAR2(20),
			 seller\_id NUMBER,
			 price NUMBER,
			 category VARCHAR2(10),
			 CONSTRAINT pk\_products PRIMARY KEY (id),
			 CONSTRAINT chk\_price CHECK (price BETWEEN 20 AND 100),
			 CONSTRAINT chk\_category CHECK (category IN ('tshirt', 'hoodie', 'jacket')),
			 CONSTRAINT fk\_product\_seller FOREIGN KEY (seller\_id) REFERENCES sellers (id)
			 ); \newline
			 
			 CREATE TABLE orders (
			 product\_id NUMBER,
			 user\_id    NUMBER,
			 CONSTRAINT pk\_orders PRIMARY KEY (product\_id, user\_id),
			 CONSTRAINT fk\_order\_product FOREIGN KEY (product\_id) REFERENCES products (id),
			 CONSTRAINT fk\_order\_user FOREIGN KEY (user\_id) REFERENCES users (id)
			 ); \newline
			 
			 CREATE TABLE order\_comments (
			 id         NUMBER,
			 product\_id NUMBER,
			 user\_id    NUMBER,
			 commenttxt    VARCHAR2(100),
			 CONSTRAINT pk\_order\_comments PRIMARY KEY (id),
			 CONSTRAINT fk\_comment\_order1 FOREIGN KEY (product\_id) REFERENCES products(id),
			 CONSTRAINT fk\_comment\_order2 FOREIGN KEY (user\_id) REFERENCES users(id) 
			 ); \newline
			 
			 CREATE TABLE seller\_reviews (
			 id        NUMBER,
			 seller\_id NUMBER,
			 user\_id   NUMBER,
			 rating    NUMBER,
			 CONSTRAINT pk\_seller\_reviews PRIMARY KEY (id),
			 CONSTRAINT chk\_rating CHECK (rating BETWEEN 1 AND 5),
			 CONSTRAINT fk\_review\_seller FOREIGN KEY (seller\_id) REFERENCES sellers (id),
			 CONSTRAINT fk\_review\_user FOREIGN KEY (user\_id) REFERENCES users (id)
			 ); \newline
			 \begin{figure}[H]
			 	\centering
			 	\includegraphics[width=0.9\textwidth]{/home/crow/Pictures/Screenshots/5.png}
			 \end{figure}
		 	 \begin{figure}[H]
		 	 \centering
		 	 \includegraphics[width=0.9\textwidth]{/home/crow/Pictures/Screenshots/6.png}
		 	 \end{figure}
	 		 \begin{figure}[H]
	 		 \centering
	 		 \includegraphics[width=0.9\textwidth]{/home/crow/Pictures/Screenshots/7.png}
	 		 \end{figure}
 		 
 		 \item 
 		 	  alter table users add lastname varchar2(20);
 		 	  \begin{figure}[H]
 		 	  	\centering
 		 	  	\includegraphics[width=0.9\textwidth]{/home/crow/Pictures/Screenshots/8.png}
 		 	  \end{figure}
 	 	  \item 
 	 	  		alter table users modify name varchar2(50) not null; \newline
 	 	  		alter table sellers modify name varcha02r(50) not null;
 	 	  		\begin{figure}[H]
 	 	  			\centering
 	 	  			\includegraphics[width=0.9\textwidth]{/home/crow/Pictures/Screenshots/9.png}
 	 	  		\end{figure}
  	  		\item 
  	  			alter table users modify name varchar2(100);
			 	\begin{figure}[H]
			 		\centering
			 		\includegraphics[width=0.9\textwidth]{/home/crow/Pictures/Screenshots/10.png}
			 	\end{figure}
		 	\item 
		 		alter table users drop column lastname;
		 		\begin{figure}[H]
		 			\centering
		 			\includegraphics[width=0.9\textwidth]{/home/crow/Pictures/Screenshots/11.png}
		 		\end{figure}
	 		\item 
	 			alter table users rename column name to fullname;
	 			\begin{figure}[H]
	 				\centering
	 				\includegraphics[width=0.9\textwidth]{/home/crow/Pictures/Screenshots/12.png}
	 			\end{figure}
 			\item 
 				alter table orders add constraint fk\_order\_seller foreign key (seller\_id) references sellers(id);
 				\begin{figure}[H]
 					\centering
 					\includegraphics[width=0.9\textwidth]{/home/crow/Pictures/Screenshots/13.png}
 				\end{figure}
 			\item 
 				alter table products add constraint dlp check(discount < price);
 				\begin{figure}[H]
 					\centering
 					\includegraphics[width=0.9\textwidth]{/home/crow/Pictures/Screenshots/14.png}
 				\end{figure}
			\item 
				CREATE OR REPLACE TRIGGER potr
				BEFORE INSERT OR UPDATE ON orders
				FOR EACH ROW
				DECLARE
				vdiscount products.discount%TYPE;
				BEGIN
				SELECT discount
				INTO vdiscount
				FROM products
				WHERE id = :NEW.product\_id;
				IF :NEW.discounted != vdiscount THEN
				RAISE\_APPLICATION\_ERROR(-20001, 'Order discount must match product discount');
				END IF;
				END;
				/
				\begin{figure}[H]
					\centering
					\includegraphics[width=0.9\textwidth]{/home/crow/Pictures/Screenshots/15.png}
				\end{figure}
			\item 
				INSERT INTO users (id, fullname) VALUES (1, 'Alice Johnson'); \newline
				INSERT INTO users (id, fullname) VALUES (2, 'Bob Smith'); \newline
				INSERT INTO users (id, fullname) VALUES (3, 'Charlie Brown'); \newline
				INSERT INTO users (id, fullname) VALUES (4, 'Diana Prince'); \newline
				INSERT INTO users (id, fullname) VALUES (5, 'Ethan Hunt'); \newline
				INSERT INTO users (id, fullname) VALUES (6, 'Fiona Apple'); \newline
				INSERT INTO users (id, fullname) VALUES (7, 'George Martin'); \newline
				
				INSERT INTO sellers (id, name) VALUES (1, 'CoolTees'); \newline
				INSERT INTO sellers (id, name) VALUES (2, 'HoodieHub'); \newline
				INSERT INTO sellers (id, name) VALUES (3, 'JacketWorld'); \newline
				INSERT INTO sellers (id, name) VALUES (4, 'UrbanWear'); \newline
				INSERT INTO sellers (id, name) VALUES (5, 'FashionFiesta'); \newline
				INSERT INTO sellers (id, name) VALUES (6, 'TrendSetters'); \newline
				INSERT INTO sellers (id, name) VALUES (7, 'StreetStyle'); \newline
				
				INSERT INTO products (id, name, seller\_id, price, category, discount) VALUES (1, 'Classic Tee', 1, 25, 'tshirt', 5); \newline
				INSERT INTO products (id, name, seller\_id, price, category, discount) VALUES (2, 'Premium Hoodie', 2, 60, 'hoodie', 10); \newline
				INSERT INTO products (id, name, seller\_id, price, category, discount) VALUES (3, 'Leather Jacket', 3, 90, 'jacket', 15); \newline
				INSERT INTO products (id, name, seller\_id, price, category, discount) VALUES (4, 'Graphic Tee', 1, 30, 'tshirt', 7); \newline
				INSERT INTO products (id, name, seller\_id, price, category, discount) VALUES (5, 'Zip Hoodie', 2, 55, 'hoodie', 12); \newline
				INSERT INTO products (id, name, seller\_id, price, category, discount) VALUES (6, 'Winter Jacket', 3, 95, 'jacket', 20); \newline
				INSERT INTO products (id, name, seller\_id, price, category, discount) VALUES (7, 'Summer Tee', 1, 22, 'tshirt', 0); \newline
				
				INSERT INTO orders (product\_id, user\_id, seller\_id, discounted) VALUES (1, 1, 1, 5); \newline
				INSERT INTO orders (product\_id, user\_id, seller\_id, discounted) VALUES (2, 2, 2, 10); \newline
				INSERT INTO orders (product\_id, user\_id, seller\_id, discounted) VALUES (3, 3, 3, 15); \newline
				INSERT INTO orders (product\_id, user\_id, seller\_id, discounted) VALUES (4, 4, 1, 7); \newline
				INSERT INTO orders (product\_id, user\_id, seller\_id, discounted) VALUES (5, 5, 2, 12); \newline
				INSERT INTO orders (product\_id, user\_id, seller\_id, discounted) VALUES (6, 6, 3, 20); \newline
				INSERT INTO orders (product\_id, user\_id, seller\_id, discounted) VALUES (7, 7, 1, 0); \newline
				
				INSERT INTO order\_comments (id, product\_id, user\_id, commenttxt) VALUES (1, 1, 1, 'Great quality!'); \newline
				INSERT INTO order\_comments (id, product\_id, user\_id, commenttxt) VALUES (2, 2, 2, 'Very cozy hoodie.'); \newline
				INSERT INTO order\_comments (id, product\_id, user\_id, commenttxt) VALUES (3, 3, 3, 'Love the jacket!'); \newline
				INSERT INTO order\_comments (id, product\_id, user\_id, commenttxt) VALUES (4, 4, 4, 'Nice graphic design.'); \newline
				INSERT INTO order\_comments (id, product\_id, user\_id, commenttxt) VALUES (5, 5, 5, 'Good value for price.'); \newline
				INSERT INTO order\_comments (id, product\_id, user\_id, commenttxt) VALUES (6, 6, 6, 'Perfect for winter.'); \newline
				INSERT INTO order\_comments (id, product\_id, user\_id, commenttxt) VALUES (7, 7, 7, 'Light and comfortable.'); \newline
				
				INSERT INTO seller\_reviews (id, seller\_id, user\_id, rating) VALUES (1, 1, 1, 5); \newline
				INSERT INTO seller\_reviews (id, seller\_id, user\_id, rating) VALUES (2, 2, 2, 4); \newline
				INSERT INTO seller\_reviews (id, seller\_id, user\_id, rating) VALUES (3, 3, 3, 5); \newline
				INSERT INTO seller\_reviews (id, seller\_id, user\_id, rating) VALUES (4, 1, 4, 3); \newline
				INSERT INTO seller\_reviews (id, seller\_id, user\_id, rating) VALUES (5, 2, 5, 4); \newline
				INSERT INTO seller\_reviews (id, seller\_id, user\_id, rating) VALUES (6, 3, 6, 5); \newline
				INSERT INTO seller\_reviews (id, seller\_id, user\_id, rating) VALUES (7, 1, 7, 4); \newline
				
				la requete "insert into orders (product\_id, user\_id, seller\_id, discounted) values (1, 2, 1, 999);" vas echouer a cause de trigger deja declaré qui necessite que discount = discounted
				 
				\begin{figure}[H]
					\centering
					\includegraphics[width=0.9\textwidth]{/home/crow/Pictures/Screenshots/16.png}
				\end{figure}
			
				la requete "INSERT INTO products (id, name, seller\_id, price, category, discount) VALUES (8, 'Expensive Tee', 1, 150, 'tshirt', 10);
				" vas echouer a cause constarint check qui necessite price>discount 
				
				\begin{figure}[H]
					\centering
					\includegraphics[width=0.9\textwidth]{/home/crow/Pictures/Screenshots/17.png}
				\end{figure}
			
				INSERT INTO products (id, name, seller\_id, price, category, discount)
				VALUES (9, 'Fancy Pants', 2, 50, 'pants', 5);
				vas echouer car la category doit etre tshirt hoodie ou jacket
				
				\begin{figure}[H]
					\centering
					\includegraphics[width=0.9\textwidth]{/home/crow/Pictures/Screenshots/18.png}
				\end{figure}
			
			\item 
				alter table products disable constraint chk\_price;
			\item 
				delete from products where id=1; \newline
				contrainte d'integrité car on a pas specifié de qui fair on delete.
				\begin{figure}[H]
					\centering
					\includegraphics[width=0.9\textwidth]{/home/crow/Pictures/Screenshots/19.png}
				\end{figure}
			\item 
				assuré
			\item 
				select * from products, orders, users where orders.product\_id = products.id and orders.user\_id = users.id and products.discount >10;
				\begin{figure}[H]
					\centering
					\includegraphics[width=0.9\textwidth]{/home/crow/Pictures/Screenshots/20.png}
				\end{figure}
			\item 
				select seller\_id , avg(discounted) as average\_discount from orders group by seller\_id;
				\begin{figure}[H]
					\centering
					\includegraphics[width=0.9\textwidth]{/home/crow/Pictures/Screenshots/21.png}
				\end{figure}
			\item 
				create view qst20 as select seller\_id , avg(discounted) as average\_discount from orders group by seller\_id;
				\begin{figure}[H]
					\centering
					\includegraphics[width=0.9\textwidth]{/home/crow/Pictures/Screenshots/22.png}
				\end{figure}
		
	\end{enumerate}

	\newpage
	
	\section{TP2}
	\begin{enumerate}
		\item 
			create user gerertp2 identified by musmus3108;
		\item 
			on peut pas connecter car l'utilisateur n'a pas le droit de creer une session
			\begin{figure}[H]
				\centering
				\includegraphics[width=0.9\textwidth]{/home/crow/Pictures/Screenshots/23.png}
			\end{figure}
		\item 
			grant create session to gerertp2;
			\begin{figure}[H]
				\centering
				\includegraphics[width=0.9\textwidth]{/home/crow/Pictures/Screenshots/24.png}
			\end{figure}
		\item 
			grant create table, create user to gerertp2;
			\begin{figure}[H]
				\centering
				\includegraphics[width=0.9\textwidth]{/home/crow/Pictures/Screenshots/25.png}
			\end{figure}
			\begin{figure}[H]
				\centering
				\includegraphics[width=0.9\textwidth]{/home/crow/Pictures/Screenshots/26.png}
			\end{figure}
			
		\item 
			
			la table order a été crée pa system donc la requete retourne que le tableau gerertp2.orders n'exxiste pas car il n y a pas une table nommé orders crée par gerertp2, d'autre par randomtable a été crée par gerertp2 donc il peut le lire.
			\begin{figure}[H]
				\centering
				\includegraphics[width=0.9\textwidth]{/home/crow/Pictures/Screenshots/27.png}
			\end{figure}
		
		\item 
			REMARQUE: JE SAIS PAS SI GERERTP2T EST UNE FAUTE DE FRAPPE OU PAS, J'ASSUME QUE LE BUTE DE LA QUESTION EST DE GERER LES DROITS D'ACCÉES DES TABLES DONC EN RESUMÉ POUR QUE USER1 PEUT FAIRE SELECT SUR UNE TABLES DE USER2 IL FAUT D'ABORD LUI DONNER LE PREVILEGE AVEC GRANT SELECT ON TABLE TO USER1 ET DEPUIS USER1 SELECT * FROM USER2.TABLE, L'EXAMPLE SUIVANT EST AVEC SYSTEM ET GERERTP2;
			
			\begin{figure}[H]
				\centering
				\includegraphics[width=0.9\textwidth]{/home/crow/Pictures/Screenshots/28.png}
			\end{figure}
			\begin{figure}[H]
				\centering
				\includegraphics[width=0.9\textwidth]{/home/crow/Pictures/Screenshots/29.png}
			\end{figure}
		\item 
			revoke select on system.orders from gerertp2; \newline
			revoke create session, create table, create user from gerertp2;
			\begin{figure}[H]
				\centering
				\includegraphics[width=0.9\textwidth]{/home/crow/Pictures/Screenshots/30.png}
			\end{figure}
		\item 
			SELECT grantee, granted\_role
			FROM dba\_role\_privs
			WHERE grantee = 'GERERTP2';
			\begin{figure}[H]
				\centering
				\includegraphics[width=0.9\textwidth]{/home/crow/Pictures/Screenshots/31.png}
			\end{figure}
		\item 
			CREATE PROFILE Gerer\_DroitTP2 \newline
			LIMIT\newline
			SESSIONS\_PER\_USER       3\newline
			CPU\_PER\_CALL            3000\newline
			CONNECT\_TIME            30\newline
			LOGICAL\_READS\_PER\_CALL  1500\newline
			PRIVATE\_SGA             25\newline
			IDLE\_TIME               40\newline
			FAILED\_LOGIN\_ATTEMPTS   3\newline
			PASSWORD\_LIFE\_TIME      80\newline
			PASSWORD\_REUSE\_TIME     60\newline
			PASSWORD\_LOCK\_TIME      1\newline
			PASSWORD\_GRACE\_TIME     25;
			
			\begin{figure}[H]
				\centering
				\includegraphics[width=0.9\textwidth]{/home/crow/Pictures/Screenshots/32.png}
			\end{figure}
		
		\item 
			alter user gerertp2 prodile gerer\_DroitTP2
			\begin{figure}[H]
				\centering
				\includegraphics[width=0.9\textwidth]{/home/crow/Pictures/Screenshots/33.png}
			\end{figure}
		\item 
			on peut pas car l'utilisateur n'a pas de previleges sur cet tables, il faut donner ce previlege aux profile
			\begin{figure}[H]
				\centering
				\includegraphics[width=0.9\textwidth]{/home/crow/Pictures/Screenshots/34.png}
			\end{figure}
		\item 
			grant alter on system.orders to gerertp2; \newline
			alter table system.orders add rn varchar(50);
			\begin{figure}[H]
				\centering
				\includegraphics[width=0.9\textwidth]{/home/crow/Pictures/Screenshots/35.png}
			\end{figure}
		\item 
			create role gestiontp2; \newline
			grant select, insert, update, delete, alter to gestiontp2;
			\begin{figure}[H]
				\centering
				\includegraphics[width=0.9\textwidth]{/home/crow/Pictures/Screenshots/36.png}
			\end{figure}
			\begin{figure}[H]
				\centering
				\includegraphics[width=0.9\textwidth]{/home/crow/Pictures/Screenshots/37.png}
			\end{figure}
			
		\item 
			grant gestiontp2 to gerertp2;
			SELECT granted\_role
			FROM dba\_role\_privs
			WHERE grantee = 'GERERTP2';
			
			\begin{figure}[H]
				\centering
				\includegraphics[width=0.9\textwidth]{/home/crow/Pictures/Screenshots/38.png}
			\end{figure}
			\begin{figure}[H]
				\centering
				\includegraphics[width=0.9\textwidth]{/home/crow/Pictures/Screenshots/39.png}
			\end{figure}
			\begin{figure}[H]
				\centering
				\includegraphics[width=0.9\textwidth]{/home/crow/Pictures/Screenshots/40.png}
			\end{figure}
			\begin{figure}[H]
				\centering
				\includegraphics[width=0.9\textwidth]{/home/crow/Pictures/Screenshots/41.png}
			\end{figure}
			
		\item 
			create index ind on sellers(name);\newline
			l'index a été crée correctement
			
			\begin{figure}[H]
				\centering
				\includegraphics[width=0.9\textwidth]{/home/crow/Pictures/Screenshots/42.png}
			\end{figure}
		\item 
			grant create any index to gerertp2;
			quand on essay de creer le meme index encore avex gerertp2 , oracle nous dit que l'index deja exist, on constate que les indexes ne sont pas isolés entre les utilisateurs
			\begin{figure}[H]
				\centering
				\includegraphics[width=0.9\textwidth]{/home/crow/Pictures/Screenshots/43.png}
			\end{figure}
				\begin{figure}[H]
				\centering
				\includegraphics[width=0.9\textwidth]{/home/crow/Pictures/Screenshots/44.png}
			\end{figure}
	\end{enumerate}

	\newpage
	
	\section{TP3}
	\begin{enumerate}
		\item 
			le catalogue DICT contient 3245 lignes et sa structure est TABLE\_NAME, COMMENTS
			\begin{figure}[H]
			\centering
			\includegraphics[width=0.9\textwidth]{/home/crow/Pictures/Screenshots/45.png}
			\end{figure}
		\item 
			1/ALL\_TAB\_COLUMNS: ça donne tout les columns que l'utilisateur courent a le droit a y acceder
			\begin{figure}[H]
				\centering
				\includegraphics[width=0.9\textwidth]{/home/crow/Pictures/Screenshots/46.png}
			\end{figure}
			2/USER\_USERS: contient des informations sure l'utilisateur courent 
			\begin{figure}[H]
				\centering
				\includegraphics[width=0.9\textwidth]{/home/crow/Pictures/Screenshots/47.png}
			\end{figure}
			3/ALL\_CONSTRAINTS: ça affiche tout les contraintes sur les tables que l'utilisateur courent peut y acceder
			\begin{figure}[H]
				\centering
				\includegraphics[width=0.9\textwidth]{/home/crow/Pictures/Screenshots/48.png}
			\end{figure}
			4/USER\_TAB\_PRIVS: affiche tout les previleges d'utilisateur courent
			\begin{figure}[H]
				\centering
				\includegraphics[width=0.9\textwidth]{/home/crow/Pictures/Screenshots/49.png}
			\end{figure}
		\item 
			select username from USER\_USERS;
			\begin{figure}[H]
				\centering
				\includegraphics[width=0.9\textwidth]{/home/crow/Pictures/Screenshots/50.png}
			\end{figure}
		\item 
			ALL\_TAB\_COLUMNS affiche tout les columns l'utilisateur a le droit a y acceder et USER\_ TAB\_COLUMNS affiche tout les columns qui appartient a l'utilisateur (owner)
		\item 
			il faut just interoger USER\_ TAB\_COLUMNS
			\begin{figure}[H]
				\centering
				\includegraphics[width=0.9\textwidth]{/home/crow/Pictures/Screenshots/53.png}
			\end{figure}
		\item 
			system a 151 tableux, notre utilisateur a 6 seulement
			\begin{figure}[H]
				\centering
				\includegraphics[width=0.9\textwidth]{/home/crow/Pictures/Screenshots/53.png}
			\end{figure}
			\begin{figure}[H]
				\centering
				\includegraphics[width=0.9\textwidth]{/home/crow/Pictures/Screenshots/52.png}
			\end{figure}
		\item 
			SELECT table\_name, column\_name, data\_type, data\_length, nullable
			FROM user\_tab\_columns
			WHERE table\_name IN ('USERS', 'PRODUCTS')
			ORDER BY table\_name, column\_id;
			
		\item 
			on peut exploiter ALL\_CONSTRAINTS
		\item 
			select constraint\_name from USER\_CONSTRAINTS ;
			\begin{figure}[H]
				\centering
				\includegraphics[width=0.9\textwidth]{/home/crow/Pictures/Screenshots/54.png}
			\end{figure}
		\item 
			on peut interoger USER\_CONSTRAINTS pour les constraintes et interoger USER\_ TAB\_COLUMNS pour avoir les noms et les types de données
		\item 
			SELECT owner, table\_name, privilege, grantable
			FROM role\_tab\_privs
			WHERE role = 'GESTIONTP2';
			\begin{figure}[H]
				\centering
				\includegraphics[width=0.9\textwidth]{/home/crow/Pictures/Screenshots/56.png}
			\end{figure}
		\item 
			SELECT grantee, granted\_role, admin\_option, default\_role
			FROM dba\_role\_privs
			WHERE grantee = 'GERERTP2';
			\begin{figure}[H]
				\centering
				\includegraphics[width=0.9\textwidth]{/home/crow/Pictures/Screenshots/55.png}
			\end{figure}
		\item 
			SELECT owner, table\_name, privilege, grantable
			FROM role\_tab\_privs
			WHERE role = 'GESTIONTP2';
			\begin{figure}[H]
				\centering
				\includegraphics[width=0.9\textwidth]{/home/crow/Pictures/Screenshots/56.png}
			\end{figure}
		\item 
			SELECT owner, table\_name
			FROM all\_tables
			WHERE table\_name = 'ORDERS';
			\begin{figure}[H]
				\centering
				\includegraphics[width=0.9\textwidth]{/home/crow/Pictures/Screenshots/57.png}
			\end{figure}
		\item
			SELECT segment\_name, bytes / 1024 AS size\_kb
			FROM user\_segments
			WHERE segment\_type = 'TABLE'
			AND segment\_name = 'ORDERS';
			\begin{figure}[H]
				\centering
				\includegraphics[width=0.9\textwidth]{/home/crow/Pictures/Screenshots/58.png}
			\end{figure}
			
	\end{enumerate}

\end{document}