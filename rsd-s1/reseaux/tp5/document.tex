\documentclass[12pt,a4paper]{report}

\usepackage{graphicx}
\usepackage{float}
\usepackage{booktabs}
\usepackage{geometry}
\geometry{
	left=2.5cm,
	right=2.5cm,
	top=2.5cm,
	bottom=2.5cm,
}

\begin{document}
	
	% ------------------------------------------------------
	%                  COVER PAGE
	% ------------------------------------------------------
	\begin{titlepage}
		\begin{center}
			
			\vspace*{2cm}
			
			{\Huge \textbf{Université de Sciences et Téchnologies Houari Boumedien}}\\[1.0cm]
			
			{\Large Faculty d'Informatique}\\[0.2cm]
			{\Large Master 1 RSD}\\[2.0cm]
			
			{\huge \textbf{TP Reseaux BGP}}\\[0.5cm]
			{\Large TP Reseaux et Protocols}\\[2.5cm]
			
			\textbf{Prof:} Mme Bouachi Farida  \\[0.5cm]
			\textbf{Etudiant:} Belmouloud Mustapha Abdellah \\[0.2cm]
			\textbf{Matricule:} 212131092524 \\[0.2cm]
			\textbf{Groupe:} 1 \\[2.5cm]
			\vfill
			
			{\large \today}
			
		\end{center}
	\end{titlepage}
	
	% ------------------------------------------------------
	%                  TABLE OF CONTENTS
	% ------------------------------------------------------
	\tableofcontents
	\newpage
	
	% ------------------------------------------------------
	%                  SECTION 1
	% ------------------------------------------------------
	\section{Réseaux EIGRP}
	
	Le Reseaux EIGRP a été créé en utilisant les commandes suivants: \\
	router eigrp 1600 \\
	network @ips\_direct @mask\_generic
	C'était la partie déja disponible a cause des tps perecedents, il ya 10 sous réseaux eigrp en total. EIGRP est un protocol de routage dynamic qui permit a les routeurs de partager les sous réseaux entre eux
	
	\section{Réseaux OSPF}
	
	\subsection{Découpage VLSM}
	avant d'appliquer le protocol ospf il faut d'abord assigner des address ip et definir nos sous réseaux, ce dernier depend largement sur la taille de chaque sous reseaux.\\
	\\
	\textbf{Zone 2 et 3: pas de decoupage nécéssaire} \\
	 \textbf{Zone 0:}\\
	 \begin{table}[H]
	 	\centering
	 	\begin{tabular}{l c c c}
	 		\toprule
	 		\textbf{nombre d'address} & \textbf{taille partie machine} & \textbf{@sous\_réseaux } & \textbf{@deffusion} \\
	 		\midrule
	 		8 & 3 & 122.12.194.128/29 & 122.12.194.135/29 \\
	 		4 & 2 & 122.12.194.136/30 & 122.12.194.139/30\\
	 		4 & 2 & 122.12.194.140/30 & 122.12.194.143/30 \\
	 		\bottomrule
	 	\end{tabular}
	 	\caption{VLSM zone 0}
	 	\label{Vlsm zone 0}
	 \end{table}
 
 	\textbf{Zone 1:}\\
 	\begin{table}[H]
 		\centering
 		\begin{tabular}{l c c c}
 			\toprule
 			\textbf{nombre d'address} & \textbf{taille partie machine} & \textbf{@sous\_réseaux } & \textbf{@deffusion} \\
 			\midrule
 			1027 & 11 & 122.12.160.0/21 & 122.12.167.255/21 \\
 			256 & 8 & 122.12.168.0/24 & 122.12.168.255/24\\
 			9 & 4 & 122.12.169.0/28 & 122.12.169.15/28 \\
 			4 & 2 & 122.12.169.16/30 & 122.12.169.19/30\\
 			4 & 2 & 122.12.169.20/30 & 122.12.169.23/30\\
 			\bottomrule
 		\end{tabular}
 		\caption{VLSM zone 1}
 		\label{Vlsm zone 1}
 	\end{table}
 
 	\subsection{Configuration OSPF}
	OSPF est un protocol de routage dynamic qui introduit le principe des zones, où zone 0 doit être directement connecté a tous autres zones, l'activation d'OSPF est trés simple, il faut just executer les commandes suivantes: \\
	router ospf 1 \\
	network @direct @mask\_generic area id\_zone
	
	\section{Réseaux BGP}
	Packet tracer ne supporte pas iBGP (BGP inetrne) donc on vas dependre sur la destribution OSPF-BGP et EIGRP-BGP. \\
	Dabord on vat activer BGP entre les deux routeur specifié \\
	
	router bgp 6506 //ou 6519 \\ 
	network 180.180.180.180 mask 255.255.255.252 \\
	neighbor @ip\_voisin remote-as 6519 //ou 6506 \\
	
	Ensuite il faut partager les sous reseaux OSPF et EIGRP avec BGP \\
	redistribute ospf 1 \\
	ou \\
	redistribute eigrp 1600 \\
	
	Ensuite il faut partager les sous réseaux reçue de BGP avec les réseaux OSPF et EIGRP \\
	
	OSPF: \\
	router ospf 1 \\
	redistribute bgp 6506 subnets \\
	
	EIGRP: \\
	router eigrp 1600 \\
	redistribute bgp 6519 metric 15050 20 255 1 1500 \\
	
	les metric sont des metric de lien comme la bande passante, il faut les inclure sinon EIGRP vas les ignorer


	\newpage

	\section{Resultas}
	
	Example avec shiw ip route
	
	\begin{figure}[H]
		\centering
		\includegraphics[width=0.9\textwidth]{1.png} % replace with your file
		\caption{Example Routeur OSPF}
		\label{fig:Router ospf}
	\end{figure}

	\begin{figure}[H]
		\centering
		\includegraphics[width=0.9\textwidth]{2.png} % replace with your file
		\caption{Example Routeur EIGRP}
		\label{fig:Router EIGRP}
	\end{figure}
	
	on remarque EX pour EIGRP et E2 pour OSPF qui indique que les reseaux sont externes
\end{document}
